\documentclass[10pt]{article}

\usepackage[utf8]{inputenc}
\usepackage[margin=0.5in]{geometry}
\usepackage{kpfonts}

\PassOptionsToPackage{usenames, dvipsnames}{xcolor}
\usepackage{figures/tikzit}

\usepackage{amsthm}
\usepackage{amssymb}
\usepackage{amsmath}
\usepackage{stmaryrd}
\usepackage{microtype}
\usepackage{enumitem}

\usepackage[style=alphabetic]{biblatex}
\usepackage[colorlinks=true, citecolor=Green, linkcolor=NavyBlue, urlcolor=BrickRed]{hyperref}
\usepackage[capitalise]{cleveref}

\input{macros/sets}
\input{macros/category}
\input{macros/streams}
\input{macros/circuits}
\input{macros/theorems}

\input{figures/circuits.tikzstyles}
\input{figures/circuits.tikzdefs}

\setcounter{biburlnumpenalty}{100}
\setcounter{biburlucpenalty}{100}
\setcounter{biburllcpenalty}{100}

\addbibresource{refs/refs.biblatex.bib}
\pagenumbering{gobble} 

\title{\vspace{-3em}Full abstraction for digital circuits: \textbf{Extended abstract}}
\author{\textbf{George Kaye}, David Sprunger and Dan R. Ghica}

\begin{document}
    \maketitle

    \paragraph*{Motivation.}
    Digital circuits are ubiquitous in today's society, so it is essential that we have easy ways to verify their correctness and reason with them.
    Conventionally, this is done by translation into an executable model such as an automaton, which can be simulated to observe its behaviour.
    An alternative approach, used commonly in software, is to reason \emph{syntactically}: programs are formulated equationally and can be reduced step by step by identifying \emph{redexes} in terms.
    When provided with inputs, the goal of such a system is to apply redexes and eventually return an output value.

    Such an equational system was first presented in~\cite{ghica2016categorical,ghica2017diagrammatic}, in which digital circuits with delay and feedback are modelled as morphisms in a freely generated traced cartesian category, or \emph{dataflow category}~\cite{cazanescu1990new,cazanescu1994feedback}.
    However, there were some informalities in the presentation, and, crucially, the framework was not guaranteed to reduce circuits with \emph{non-delay-guarded feedback} to a value.

    The aim of this work is to wrap up the loose ends of the and present a sound and complete axiomatic framework for handling \emph{any} circuit with delay and feedback.

    \paragraph*{Syntax.}

    \begin{definition}[Circuit signature]
        Let \(\Sigma\) be a tuple \((\mathcal{V},\mathcal{G})\) where \(\mathcal{V}\) is a finite set of \emph{values} and \(\mathcal{G}\) is a finite set of tuples \((g,m)\) where \(g\) is a \emph{gate symbol} and \(m \in \nat\) is its \emph{arity}.
    \end{definition}

    \begin{definition}[Sequential circuits]
        For a signature \(\Sigma\), let \(\scircsigma\) be the symmetric traced monoidal category freely generated over:
        \begin{align*}
            \iltikzfig{strings/structure/monoid/init} 
            &&
            \iltikzfig{strings/structure/monoid/init-white} 
            &&
            \iltikzfig{circuits/components/values/v}\ \text{for each}\ v \in \circuitsignaturevalues
            &&
            \iltikzfig{circuits/components/gates/gate-labelled}\ \text{for each}\ (g,m) \in \circuitsignaturegates
            &&
            \iltikzfig{strings/structure/comonoid/copy}
            &&
            \iltikzfig{strings/structure/monoid/merge}
            &&
            \iltikzfig{strings/structure/comonoid/discard}
            &&
            \iltikzfig{circuits/components/delay}
        \end{align*}
    \end{definition}    

    \noindent
    The small boxes are \emph{values}: these represent the signals that can flow through our circuits.
    The first two are special: \iltikzfig{strings/structure/monoid/init} represents a \emph{disconnected wire}: a \emph{lack} of information.
    Conversely, \iltikzfig{strings/structure/monoid/init-white} represents a \emph{short circuit}: a \emph{glut} of information.
    The remaining values are specified by our circuit signature.
    Next come the generators for each gate symbol in our signature.
    Then there are three \emph{structural} generators for forking, joining and stubbing wires.
    The final generator is a \emph{delay} generator: one can think of this as delaying its inputs for one tick of the clock.
    We write arbitrary sequential circuits obtained by composing generators together as squares \iltikzfig{circuits/components/circuits/f-seq}.
    If a circuit is \emph{combinational}, i.e. it contains no delay or trace, it is drawn in a lighter square \iltikzfig{circuits/components/circuits/f-comb}.

    \paragraph*{Semantics.}

    Circuits specified syntactically have no computational content associated with them.
    To add \emph{semantics} to circuits, first the signature must be interpreted in some domain.

    \begin{definition}[Interpretation]\label{def:interpretation}
        Let \(\circuitsignature = (\circuitsignaturevalues,\circuitsignaturegates)\) be a circuit signature.
        A interpretation of \(\circuitsignature\) is a tuple \(\interpretation = (\mathbf{V}, \valueinterpretation,\gateinterpretation)\) where \(\values\) is a finite lattice, \(\valueinterpretation\) is a bijective function \(\valueinterpretation \to \values \setminus \{\bot,\top\}\) and \(\gateinterpretation\) is a map from each gate \((g,m) \in \circuitsignaturegates\) to a monotone function \(\morph{\overline{g}}{\valuetuple{m}}{\values}\).
    \end{definition}

    \noindent
    The semantics of circuits is that of \emph{stream functions}, which take as input a stream and output a stream.
    In particular, we are interested in stream functions of the form \(\valuetuplestream{m} \to \valuetuplestream{n}\).

    \begin{definition}
        For an interpretation \(\interpretation = (\values, \valueinterpretation, \gateinterpretation)\), let \(\streami\) be the prop with morphisms \(m \to n\) as stream functions \(\valuetuplestream{m} \to \valuetuplestream{n}\) freely generated over
        stream functions for values \(\morph{\tilde{v}}{0}{1}\) for each \(v \in \values\), defined as \(\tilde{v}(0) = v\) and \(\tilde{v}(i) = \bot\); for gates \(\morph{\tilde{g}}{m}{1}\) for each \((g,m) \in \circuitsignaturegates\) defined as \(\tilde{g}(\sigma)(i) = g(\sigma(i))\); and for delay \(\morph{\delay}{1}{1}\) defined as \(\delay(\sigma)(0) = \bot\) and \(\delay(\sigma)(i+1)\).
    \end{definition}

    \begin{theorem}
        \(\streami\) is traced.
    \end{theorem}

    \begin{definition}\label{def:circuittostreams}
        Let \(\morph{\circuittostream{\interpretation}}{\scircsigma}{\streami}\) be a traced prop morphism, mapping circuits to appropriate stream functions.
        The details are omitted, see~\cite{ghica2022full}.
    \end{definition}

    \noindent
    If two circuits map to the same semantics in \(\streami\), we say they are \emph{extensionally equivalent}, written \(\iltikzfig{circuits/components/circuits/f-seq} \extequivi \iltikzfig{circuits/components/circuits/g-seq}\).

    \begin{theorem}[\cite{ghica2022full}]
        Let \(\scircsigmai\) be the category obtained by quotienting \(\scircsigma\) by \(\extequivi\).
        Then, \(\scircsigmai \cong \streami\).
    \end{theorem}

    \paragraph*{Equational reasoning.}

    Circuits of different structure can have the same semantics.
    \begin{example}
        The circuits \iltikzfig{circuits/examples/demorgan-lhs} and \iltikzfig{circuits/examples/demorgan-rhs} are syntactically different.
        However, under the interpretation \(\{\iltikzfig{circuits/components/gates/and} \mapsto \wedge, \iltikzfig{circuits/components/gates/or} \mapsto \vee, \iltikzfig{circuits/components/gates/not} \mapsto \neg\}\), the two circuits map to the same stream function (by applying de Morgan's law).
    \end{example}

    However, in general it is computationally prohibitive to check that the corresponding streams for two circuits are equal~\cite{ghica2017diagrammatica}.
    It is far more efficient to reason \emph{equationally}.
    
    Given a set of axioms \(\mathcal{E}\), we write \(\iltikzfig{circuits/components/circuits/f-seq} \eqaxioms{\mathcal{E}} \iltikzfig{circuits/components/circuits/g-seq}\) if \iltikzfig{circuits/components/circuits/f-seq} can be rewritten to \iltikzfig{circuits/components/circuits/g-seq} by equations in \(\mathcal{E}\).

    \begin{theorem}
        Given values 
    \end{theorem}
    \begin{proof}
        
    \end{proof}


    \begin{figure}
        \newcommand{\circaxiom}[2]{\iltikzfig{circuits/axioms/#1-#2}}
        \newcommand{\structaxiom}[2]{\iltikzfig{strings/structure/#1-#2}}
        \newcommand{\cartaxiom}[2]{\iltikzfig{strings/cartesian/#1-#2}}
        \centering
        \begin{align*}
            \circaxiom{fork}{lhs}
            &\eqaxiomsc
            \circaxiom{fork}{rhs}
            &
            \circaxiom{join}{lhs}
            &\eqaxiomsc
            \circaxiom{join}{rhs}
            &
            \circaxiom{gate}{lhs}
            &\eqaxiomsc
            \circaxiom{gate}{rhs}
        \end{align*}
        \caption{The set of axioms \(\mathcal{C}\) for reducing closed circuits.}
    \end{figure}


    \paragraph*{Productivity.}

    \begin{theorem}
        For any closed sequential circuit \iltikzfig{circuits/components/circuits/f-seq-closed}, there exist values \iltikzfig{circuits/components/values/vs} and sequential circuit \iltikzfig{circuits/components/circuits/g-seq-closed} such that \(\iltikzfig{circuits/components/circuits/f-seq-closed} \eqaxiomsc \iltikzfig{circuits/productivity/productive}\).
    \end{theorem}

    \printbibliography[heading=bibintoc,title={References}]

\end{document}
